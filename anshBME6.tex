\documentclass[12pt]{article}

\usepackage{titlepic}
\usepackage{graphicx}

\usepackage{adjustbox}
\graphicspath{{images_anshBME/}}

\usepackage{hyperref}
\hypersetup{colorlinks = true, citecolor = blue, linkcolor = blue, urlcolor = blue}

\title{Basic Biomedical Engineering\\Assignment 6\\.\\Contribution of Biomedical Engineers in fight against COVID-19}

\titlepic{\includegraphics[scale=1]{logo.jpg}}

\author{Submitted by: \\Name:- Ansh Shrivastava\\Email:- anshshrivastava29@gmail.com\\Roll No :- 21111011\\First Semester, Biomedical Engineering\\.\\Under the supervision of\\Saurabh Gupta Sir}
%\date{January 2022}


\begin{document}
\maketitle
\clearpage
\tableofcontents
\clearpage

\section{Solutions to COVID-19 provided by Biomedical Engineers}
\subsection{Overview}
COVID-19 is one of the most severe global health crises that humanity has ever faced.From the Thermometer measuring Fever to the ventilators helping keep patients alive, Biomedical engineers have a big role to play in humanity's efforts to fight COVID-19. 
Many people working in engineering have responded to the ongoing crisis by adapting their existing skills and equipment to help fight COVID-19. The role of a Biomedical Engineer includes designing biomedical equipment and devices to aid the recovery or improve the health of individuals.
 
\subsection{Rapid POC and home-based testing/diagnostics}
Home based testing kits for corona virus plays a vital role in determining the presence of infection when in the first time. This can roughly confirm that you have corona infection or not. This is a quick testing process which can give almost immediate result to check your infection.
Home testing kits for COVID-19 which is based on saliva can be much more effective because that is how official testing is taking place. It may be possible that person going through that test may become negative in test, so it is recommended to test it once again on a gap of 3-4 days.

\subsection{Digital health platforms and models that integrate data, Management Tools}
Big data and artificial intelligence (AI) have helped 
facilitate COVID-19 preparedness and the tracking of 
people, and so the spread of infection, in several 
countries. Tools such as migration maps, which use 
Mobile Phones Network, GPS, and Social 
Media to collect real-time data on the location of people.
\subsection{Technologies for protecting healthcare workers, first responders, and caregivers}
 The protection of health care workers is vital in continuing patient care in health care systems that are currently challenged by the pandemic, but also important in ensuring they do not spread the virus. so, access to personal protective equipment (PPE) for health workers is a key concern. Therefore PPE Kit, Medical Masks and Gloves are Manufactured in a Large Scale across the Globe to Help in Fighting with Pandemic.

\subsection{Oxygen Therapy}
The delivery of additional oxygen using a nasal cannula or a more intrusive face mask is usually the primary form of treatment for mild respiratory insufficiency. The oxygen is usually delivered in cylinders, which are either tiny for transportation or big for fixed patients and longer-term supplies.Although oxygen concentrators are an appealing option to tanks, they are rarely used in hospital settings for caring for COVID-19 patients. Oxygen concentrators take oxygen from the air and deliver it to the patient on demand. Concentrators are available in a variety of sizes, ranging from a small portable shoulder bag to larger fixed units for patients who require oxygen 24 hours a day.High flow nasal oxygen (HFNO) is a type of oxygen delivery that delivers warmed and humidified oxygen at high flow rates (usually tens of litres/min) at body temperature and up to 100 percent RH and 100 percent oxygen to avoid drying out the airways.
\subsection{Ventilators}
Patients who cannot breathe spontaneously need to be put on a ventilator. Ventilators are capable of replacing the breath function and patients in an advanced state of respiratory distress are usually intubated and sedated at the beginning of the treatment.\newline
Ventilators are capable of replacing the breath function and patients in an advanced state of respiratory distress are usually intubated and sedated at the beginning of the treatment. They are complex systems providing the healthcare professionals with a lot of flexibility to adapt the assisted breathing settings and to be able to wean recovering patients off the ventilator gradually.
\end{document}