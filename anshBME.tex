\documentclass[12pt]{article}

\usepackage{titlepic}
\usepackage{graphicx}

\usepackage{adjustbox}
\graphicspath{{images_anshBME/}}

\usepackage{hyperref}
\hypersetup{colorlinks = true, citecolor = blue, linkcolor = blue, urlcolor = blue}

\title{Basic Biomedical Engineering\\Assignment 1\\.\\Write-up on 5 Medical Devices}

\titlepic{\includegraphics[scale=1]{logo.jpg}}

\author{Submitted by: \\Name:- Ansh Shrivastava\\Email:- anshshrivastava29@gmail.com\\Roll No :- 21111011\\First Semester, Biomedical Engineering\\.\\Under the supervision of\\Saurabh Gupta Sir}
%\date{January 2022}


\begin{document}
\maketitle
\clearpage
\tableofcontents
\clearpage

\section{Otoscope}
\subsection{Overview}
An Otoscope is a medical equipment which is used to look into the ears. It is also known as Auriscope.Otoscopes are used by doctors to check for sickness during routine check-ups and to analyse ear complaints. An Otoscope can provide a view of the ear canal and tympanic membrane, often known as the eardrum. Because the eardrum is the boundary between the external ear canal and the middle ear, its properties can indicate a variety of middle ear illnesses. Earwax (cerumen), shed skin, pus, canal skin edema, foreign bodies, and numerous ear disorders can obscure any view of the eardrum, reducing the value of otoscopy performed with a standard otoscope.
\\
\begin{figure}[h]
\centering
\includegraphics[scale=1]{otoscope1.jpg}
\label{fig_otoscope1}
\end{figure}
\\
\begin{figure}[h]
\centering
\includegraphics[scale=1]{otoscope2.jpg}
\label{fig_otoscope2}
\end{figure}
\\
\subsection{Compositon of a general Otoscope}
The most popular otoscopes have two parts: a handle and a head. A light source and a simple low-power magnifying lens, usually approximately 8 diopters, are included in the head. An attachment for disposable plastic ear specula is located on the otoscope's distal (front) end. The examiner straightens the ear canal by tugging on the pinna (typically the earlobe, side, or top) before inserting the otoscope's ear speculum side into the external ear. To avoid harm to the ear canal, brace the hand holding the otoscope against the patient's head by placing the index or little finger against the head.
\\
\subsection{Otoscopy}
Otoscopy is a medical treatment that involves looking inside the ear, specifically the external auditory canal, tympanic membrane, and middle ear with the help of an Otoscope. The provider holds the otoscope's handle and inserts the otoscope's cone into the patient's external auditory canal while doing the otoscopic examination. The otoscope has a light and a magnifying lens that illuminates and enlarges ear structures, allowing the provider to see and evaluate the health of the visible anatomical structures.

\subsection{Process of examining the ear}
Examiner first washes his/her hand and then explain the procedure to the patient. Then the examiner obtains the patient's consent.\\
Firstly the examiner asks the patient about their better hearing ear and whether they have got any pain or tenderness.\\
Now the examiner asks the patient to give them a smile. By asking the patient to smile the examiner looks for facial weakness, which is one of the five cardinal symptoms of ear disease. The facial nerve runs in the medial wall of the middle ear and may be effective by pathology in the ear.
\\
\begin{figure}[h]
\centering
\includegraphics[scale=1]{otoscope3.jpg}
\label{fig_otoscope3}
\end{figure}
\\
The Examiner starts by examining the better hearing ear. First they inspect the external ear using the light of the otoscope to help, look at the color of the skinand the shape of the pinner, then they look for other abnormalities such as toe fie and congenital defects.\\
Next otoscopy is performed to visualize the external auditory meatus and the tympanic membrane. When examining the right ear hold the otoscope in your right hand, hold it like a pen balanced between your thumb and index finger. extend your little finger and use this to rest against the patient's face, this means if the patient moves his head, the examiner's hand on the otoscope will move with it and the otoscope will not stab in the ear canal and cause pain, this is of particular importance when examining children with your free left hand pull the pin gently upwards, outwards and backwards to straighten the external ear canal to allow better view tympanic membrane. In children pull the pin down and back instead.\\
Insert the otoscope into the external auditory meatus and inspect the canal for wax discharge information and other abnormalities. Gently proceed down the canal until the tympanic membrane is seen. Move the speculum around to examine the whole of the tympanic membrane and Identify the normal landmarks.
%Sir Please iske extra marks consider kijiyega, khud se likha hai poora :)

\section{Electrocardiogram (ECG)}
\subsection{Overview}
An electrocardiogram (ECG) is a simple test that checks the rhythm and electrical activity of your heart.The electrical signals produced by your heart each time it beats are detected by sensors placed to your skin.
A machine records these signals, which a doctor examines to see if they are odd.An ECG may be requested by a cardiologist or any doctor who suspects you have a cardiac condition, including your general practitioner.A professionally qualified healthcare expert can perform the test in a hospital, clinic, or at your GP's office.
\\
\begin{figure}[h]
\centering
\includegraphics[scale=1]{ecg1.jpg}
\label{fig_ecg1}
\end{figure}
\\
A healthy heart has an orderly progression of depolarization during each heartbeat that begins with pacemaker cells in the sinoatrial node, spreads throughout the atrium, and passes through the atrioventricular node down into the bundle of His and into the Purkinje fibres, spreading down and to the left throughout the ventricles. The classic ECG tracing is created by this organised pattern of depolarization. An ECG can be used to assess the rate and rhythm of heartbeats, the size and position of heart chambers, the presence of any damage to the heart's muscle cells or conduction system, the effects of heart medications, and the performance of pacemakers, among other things.
\\
\subsection{Components of ECG}
Ten electrodes are put on the patient's extremities and on the chest surface in a traditional 12-lead ECG. The overall magnitude of the heart's electrical potential is then measured and recorded over time from twelve different angles ("leads") (usually ten seconds). Throughout the cardiac cycle, the total amount and direction of the heart's electrical depolarization are captured in this way.
\\
The P wave, which represents the depolarization of the atria; the QRS complex, which represents the depolarization of the ventricles; and the T wave, which represents the repolarization of the ventricles, are the three primary components of an ECG.
\\
\subsection{Medical uses of ECG}
The purpose of an ECG is to learn about the electrical function of the heart. This information has a wide range of medical applications, and it is frequently paired with knowledge of the heart's structure and physical examination signs to be interpreted. Short intermittent tracings or continuous ECG monitoring are also options for recording ECGs. Critically sick patients, patients undergoing general anesthesia and individuals with an infrequently occurring cardiac arrhythmia that would be difficult to detect on a traditional ten-second ECG are all candidates for continuous monitoring.
\clearpage

\section{Sphygmomanometer (Blood pressure monitor)}
\subsection{Overview}
A sphygmomanometer, also known as a blood pressure monitor or blood pressure gauge, is a blood pressure measuring device that consists of an inflatable cuff that is used to collapse and then release the artery under the cuff in a controlled manner, as well as a mercury or aneroid manometer to measure the pressure. When employing the auscultatory approach, manual sphygmomanometers are utilised in conjunction with a stethoscope.
An inflating cuff, a measurement device (the mercury manometer, or aneroid gauge), and an inflation mechanism (either a manually driven bulb and valve or an electrically operated pump) make up a sphygmomanometer.
\\
\subsection{Digital Blood Pressure Monitor}
Instead of auscultation, digital metres use oscillometric measurements and electronic calculations. They can inflate manually or automatically, but both are electronic, simple to use without training, and may be utilised in noisy conditions. They use oscillometric detection to measure systolic and diastolic pressures, and they have a microprocessor. They use deformable membranes that are measured using differential capacitance or differential piezoresistance, and they have a microprocessor. They measure mean blood pressure and pulse rate, although systolic and diastolic pressures are acquired with less precision than manual metres, and calibration is an issue.
\\
\begin{figure}[h]
\centering
\includegraphics[scale=1]{bp1.jpg}
\label{fig_bp1}
\end{figure}
\\
\subsection{Operation}
The cuff is generally put smoothly and snugly around an upper arm while the subject is seated with the arm supported, at roughly the same vertical height as the heart. A pressure that is too high is caused by a cuff that is too tiny, while a pressure that is too low is caused by a cuff that is too large. In order to assess if one arm's pressure is significantly greater than the other, clinicians typically measure and record both arms during the initial consultation. A difference of 10 mm Hg could indicate aortic coarctation. The higher reading arm would be used for later readings if the arms read differently. The cuff is inflated until it fully obstructs the artery.
\\
\subsection{Significance}
The operator records the blood pressure in mm Hg by watching the mercury in the column while releasing the air pressure with a control valve. The systolic pressure is the highest pressure in the arteries during the cardiac cycle, while the diastolic pressure is the lowest pressure (during the cardiac cycle's resting phase). In the auscultatory approach, a stethoscope is delicately placed over the artery being measured. The first of the continuous Korotkoff noises identifies systolic pressure (initial phase). When the Korotkoff noises stop, the diastolic pressure is measured (fifth phase). Blood pressure is measured in the diagnosis and treatment of hypertension (high blood pressure), as well as in a variety of other medical situations.
\\
\clearpage
\section{Thermometer}
\subsection{Overview}
A thermometer is a device that detects temperature or a temperature difference (the degree of hotness or coldness of an object). A thermometer has two important components: (1) a temperature sensor (e.g., the bulb of a mercury-in-glass thermometer or the pyrometric sensor in an infrared thermometer) that changes when the temperature changes, and (2) a means of converting this change into a numerical value (e.g. the visible scale that is marked on a mercury-in-glass thermometer or the digital readout on an infrared model). Thermometers are frequently used to monitor processes in technology and industry, as well as in meteorology, medicine, and scientific study.
\\
\subsection{Precision}
A thermometer's accuracy or resolution refers to how accurate it can read to a fraction of a degree. When working at high temperatures, it's probable that you'll only be able to measure to the nearest 10 °C or more. Clinical thermometers and many electronic thermometers can read temperatures down to 0.1 degrees Celsius. Special devices can provide readings down to the thousandth degree.
\\
\subsection{Applications of Thermometer}
Nanothermometry is a new study topic that deals with temperature measurement on a sub-micrometric scale. Traditional thermometers are unable of measuring the temperature of an item smaller than a micrometre, necessitating the employment of innovative technologies and materials. In such instances, nanothermometry is used. Luminescent thermometers (those that utilise light to measure temperature) and non-luminescent thermometers are two types of nanothermometers (systems where thermometric properties are not directly related to luminescence).
\\
Cryometer, Thermometers used specifically for low temperatures.
\\
Medical, Infrared thermometers are commonly used in ear thermometers.
A liquid crystal thermometer is an example of a forehead thermometer.
Rectal and oral thermometers used to be mercury-based, but NTC thermistors with a digital reading have generally replaced them. Various thermometric techniques, such as the Galileo thermometer and thermal imaging, have been utilised throughout history. In health care settings, medical thermometers such as mercury-in-glass thermometers, infrared thermometers, pill thermometers, and liquid crystal thermometers are used to identify if patients have a fever or are hypothermic. Food quality and safety
\\
\begin{figure}[h]
\centering
\includegraphics[scale=1]{thermo1.jpg}
\label{fig_thermo1}
\end{figure}
\clearpage
\section{Pulse Oximeter}
\subsection{Overview}
Pulse oximetry is a noninvasive technique for measuring an individual's oxygen saturation. The more accurate (and intrusive) assessment of arterial oxygen saturation (SaO2) from arterial blood gas analysis is usually within 2 percent accuracy (within 4 percent accuracy in 95 percent of cases). [1] However, the two are sufficiently connected that the pulse oximetry approach is useful for detecting oxygen saturation in clinical settings. It is safe, convenient, noninvasive, and affordable.
\\
\\
Transmissive pulse oximetry is the most prevalent method. A sensor device is placed on a thin part of the patient's body, typically a fingertip or earlobe, or an infant's foot, in this procedure. The blood flow rates in fingertip and earlobes are higher than in other tissues, allowing for better heat transfer. Two wavelengths of light are sent via the body part to a photodetector by the gadget. It determines the absorbances attributable to pulsing arterial blood alone, eliminating venous blood, skin, bone, muscle, fat, and (in most cases) nail polish, by measuring the changing absorbance at each of the wavelengths.

\subsection{Application}
A pulse oximeter is a medical device that produces a photoplethysmogram, which can be subsequently processed into various data, by indirectly monitoring the oxygen saturation of a patient's blood (rather than directly measuring oxygen saturation through a blood sample).
\\
\\
Pulse oximetry is particularly useful for noninvasive continuous blood oxygen saturation measurements. Blood gas levels, on the other hand, must be tested in a laboratory using a taken blood sample. Pulse oximetry is useful for assessing any patient's oxygenation and determining the effectiveness of or need for supplemental oxygen in any setting where a patient's oxygenation is unstable, such as intensive care, operating, recovery, emergency, and hospital ward settings, as well as pilots in unpressurized aircraft. A pulse oximeter can monitor oxygenation, but it can't tell you how much oxygen a patient is using or how much oxygen he or she is using. Carbon dioxide (CO2) levels must also be measured for this reason.
\\
\begin{figure}[h]
\centering
\includegraphics[scale=1]{oxi1.jpg}
\label{fig_oxi1}
\end{figure}
\\
\subsection{Working}
The proportion of blood filled with oxygen is displayed on a blood-oxygen monitor. It determines what percentage of haemoglobin, the oxygen-carrying protein in the blood, is loaded. For patients without pulmonary disease, normal SaO2 ranges from 95 to 99 percent. [requires citation] The "saturation of peripheral oxygen" (SpO2) value on a blood-oxygen monitor can be used to determine arterial pO2 for a person breathing room air at or near sea level.


\end{document}

