\documentclass[12pt]{article}

\usepackage{titlepic}
\usepackage{graphicx}

\usepackage{adjustbox}
\graphicspath{{images_anshBME/}}

\usepackage{hyperref}
\hypersetup{colorlinks = true, citecolor = blue, linkcolor = blue, urlcolor = blue}

\title{Basic Biomedical Engineering\\Assignment 5\\.\\Emerging Technologies in Healthcare}

\titlepic{\includegraphics[scale=1]{logo.jpg}}

\author{Submitted by: \\Name:- Ansh Shrivastava\\Email:- anshshrivastava29@gmail.com\\Roll No :- 21111011\\First Semester, Biomedical Engineering\\.\\Under the supervision of\\Saurabh Gupta Sir}
%\date{January 2022}


\begin{document}
\maketitle
\clearpage
\tableofcontents
\clearpage

\section{Emerging Technologies in Healthcare}
\subsection{Overview}
Technological innovations have a huge impact on today's healthcare industry. Emerging technologies are assisting in the development of novel, better treatments while reducing costs. Some technologies have yet to be fully explored, but they have already caused a significant shift in the industry. Artificial intelligence and robotics, for example, are fundamentally altering the landscape and ushering in a new era in healthcare.
\subsection{Portable diagnostics}
Portable diagnostics refers to a group of instruments that can be used to diagnose or monitor disease both inside and outside of hospitals. The instruments available in this sector are practically limitless, ranging from monitoring the chemistry of physiological fluids to collecting real-time photographs of our inside organs, and the list of possibilities continues to grow. Stethoscopes, thermometers, and point-of-care ultrasonography equipment are some of the most frequent tools used by medical personnel. Patients frequently use at-home pregnancy tests and continuous glucose monitors, both of which are classified as 'portable diagnostics'.
\subsection{Nutrigenomics}
The study of the impact of food and food ingredients on gene expression, as well as how genetic differences affect the nutritional environment, is known as nutrigenomics. It focuses on determining how nutrients and other dietary bioactives interact with the genome at the molecular level in order to determine how certain nutrients or dietary regimens may affect human health.
\subsection{3D Bioprinting}
3D bioprinting is another technological advancement in the healthcare field. By 2027, the worldwide bioprinting market might be worth over USD 1.8 billion. Bio-printing can restore and replace many body parts, bones, and tissue using DNA analysis. A study team recently devised a way for printing biological skin and blood vessels in 3D. This is a huge step forward for burn victims who need skin grafts. Patients who have lost limbs will benefit from 3D-printed prosthesis.
\subsection{Robotics}
Robotics has also advanced significantly, making it a top healthcare technology trend. Surgeons can use medical robots to perform extremely accurate and targeted treatments and cures. While doctors still have control over the surgery, robots eliminate the danger of human error and may help to prevent infection. Nursing and other healthcare staff will be able to focus more on direct patient care as robots take over clerical and routine jobs.
\subsection{Artificial intelligence}
AI is revolutionising how healthcare organisations handle and derive insights from the massive amounts of scientific data and patient data that are now available. AI can be used to design and customise treatment regimens and pharmaceutical options for patients far more quickly and accurately than human healthcare teams can. AI can also assist in other areas, such as developing genomic medicine by analysing complex genetic data to decide the best course of treatment for individuals based on their DNA. AI may one day increase diagnosis accuracy and even predict health consequences, according to the hope.
\end{document}