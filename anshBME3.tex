\documentclass[12pt]{article}

\usepackage{titlepic}
\usepackage{graphicx}

\usepackage{adjustbox}
\graphicspath{{images_anshBME/}}

\usepackage{hyperref}
\hypersetup{colorlinks = true, citecolor = blue, linkcolor = blue, urlcolor = blue}

\title{Basic Biomedical Engineering\\Assignment 3\\.\\Future of Healthcare}

\titlepic{\includegraphics[scale=1]{logo.jpg}}

\author{Submitted by: \\Name:- Ansh Shrivastava\\Email:- anshshrivastava29@gmail.com\\Roll No :- 21111011\\First Semester, Biomedical Engineering\\.\\Under the supervision of\\Saurabh Gupta Sir}
%\date{January 2022}


\begin{document}
\maketitle
\clearpage
\tableofcontents
\clearpage

\section{Future of Healthcare}
\subsection{Overview}
Artificial intelligence, virtual reality, 3D printing, robots, and nanotechnology, among other digital healthcare technologies, are changing the future of healthcare right before our eyes. To be able to influence technology rather than the other way around, we must stay up with current events. The future of healthcare is to work hand in hand with technology, and healthcare workers must embrace modern healthcare technologies to stay relevant in the next years.
\\
\\
Digital technology in medicine and healthcare could help transform unsustainable healthcare systems into sustainable ones, equalise the relationship between medical professionals and patients, provide cheaper, faster, and more effective disease solutions – technologies could help us win the battle against cancer, AIDS, and Ebola – and simply lead to healthier individuals living in healthier communities.
\\
This article will look at five ways that medical technology is changing the healthcare industry.
\subsection{Artificial intelligence}
AI algorithms can mine medical records, build treatment plans, and create medications more faster than any other actor in the healthcare ecosystem, including doctors. Google's DeepMind recently developed an A.I. for breast cancer analysis. On pre-selected data sets, the algorithm outscored all human radiologists by 11.5 percent on average in detecting breast cancer. From developing new pharmaceuticals to disrupting medical imaging to mining medical information, artificial intelligence will advance healthcare.
\subsection{Virtual reality}
Virtual reality (VR) is transforming the lives of both patients and doctors. In the future, you might be able to witness procedures as if you were the surgeon, or you might be able to go to Iceland or back home while laying in a hospital bed.Virtual reality is being utilised to train future surgeons as well as to practise surgeries by current surgeons. Companies like Osso VR and ImmersiveTouch have developed and delivered such software programmes, which are currently in use with promising results. According to a recent Harvard Business Review research, virtual reality-trained surgeons outperformed their traditional-trained colleagues by 230 percent. In addition, the former were speedier and more precise when executing surgical procedures.
\subsection{Augmented reality}
Augmented reality varies from virtual reality in two ways: users don't lose connection with reality, and information is delivered as quickly as feasible. These distinguishing characteristics allow AR to become a driving force in the future of medicine, both for healthcare providers and for patients. In the case of medical practitioners, it may assist medical students better prepare for real-life surgeries while also allowing surgeons to improve their skills. Students at Case Western Reserve University are already utilising the Microsoft HoloLens to study anatomy through the HoloAnatomy app. Medical students can learn the subject without using real bodies by using this technology, which gives them access to detailed and precise, but virtual, models of the human anatomy.
\subsection{Healthcare trackers, wearables and sensors}
They're fantastic tools for learning more about ourselves and regaining control of our life. There is a device for all of these needs and more! Whether you want to better manage your weight, stress, cognitive abilities, or overall fitness and energy, there is a device for all of these needs and more! The beauty of these new technology-driven gadgets is that they truly put patients at the centre of care. These technologies enable people to take charge of their health and make more informed decisions by allowing them to monitor their health at home and communicate the data with their physician remotely.
\subsection{3D-printing}
3D printing has the potential to revolutionise healthcare in every way. We can now print biotissues, artificial limbs, medications, blood vessels, and the list goes on, and it is likely that we will continue to do so in the future. Rensselaer Polytechnic Institute in Troy, New York, found a technology for printing living skin and blood vessels in 3D. This breakthrough is critical for burn victims who require skin grafts. This technique also benefits the pharmaceutical business. Since 2015, the FDA has approved 3D-printed medications, and researchers are now working on 3D-printed "polypills." These contain multiple layers of medications to aid patients in sticking to their treatment plans.
\end{document}