\documentclass[12pt]{article}

\usepackage{titlepic}
\usepackage{graphicx}

\usepackage{adjustbox}
\graphicspath{{images_anshBME/}}

\usepackage{hyperref}
\hypersetup{colorlinks = true, citecolor = blue, linkcolor = blue, urlcolor = blue}

\title{Basic Biomedical Engineering\\Assignment 4\\.\\Disruptive Innovations in Healthcare}

\titlepic{\includegraphics[scale=1]{logo.jpg}}

\author{Submitted by: \\Name:- Ansh Shrivastava\\Email:- anshshrivastava29@gmail.com\\Roll No :- 21111011\\First Semester, Biomedical Engineering\\.\\Under the supervision of\\Saurabh Gupta Sir}
%\date{January 2022}


\begin{document}
\maketitle
\clearpage
\tableofcontents
\clearpage

\section{Disruptive Innovations in Healthcare}
\subsection{Overview}
Disruptive innovations are those that bring about significant change and frequently result in the emergence of new industry leaders. They overturn the usual way of doing things to such an extent that they have a ripple effect throughout the industry,  and technology is the primary engine of many disruptive innovations in healthcare, as every area of the industry is reliant on some sort of technology. The following are five examples of disruptive innovations in healthcare which are centered on technology
\subsection{Drones - In Healthcare Delivery}
Drones have shown remarkable promise in a variety of healthcare settings. Drones for telehealth and virtual care have been one of the most significant applications. Drones are also being used to deliver healthcare supplies and goods all around the world. Drones might transport medications and supplies to patients who are being cared for at home rather than at a hospital. Outpatient and even home-based care will become increasingly common in the future, replacing hospital-based care. Drone technology could make providing home-based care easier and safer in many cases.
\\
\begin{figure}[h]
\centering
\includegraphics[scale=1]{droneHealthcare.jpg}
\label{fig_droneHealthcare}
\end{figure}
\\
\subsection{Mobile Health Technology}
Mobile health technology, or mHealth, is a fast evolving aspect in today's health care, with the potential to improve and streamline care. According to a recent poll, 83 percent of U.S. physicians utilise mobile health technology to offer patient care. The use of smartphones, tablets, and other mobile devices to deliver health care and preventive health services is known as mobile health technology. Access clinical information (e.g., through mobile health apps and mobile-enabled EHRs), collaborate with care teams (e.g., through secure text messaging), communicate with patients (e.g., through patient portals), provide real-time patient monitoring, and provide health care remotely (also known as telemedicine) are all examples of how health care providers use mobile health technology.
\subsection{Artificial Intelligence}
Artificial Intelligence plays a role in reimagining healthcare and disruptive healthcare innovations, like the AI called Babylon Health, where you get to log on anytime and within a minute talk to a doctor and the doctor is interacting with you just like a video call on Skype or zoom but he has access to your medical records and the AI is doing two things - one it's looking at your face and detects that the patient looks worried and feeds it to the doctor that it sounds like a serious condition,  the other thing the AI is doing is going through the thousands of decision trees which help the doctor diagnose the issue more accurately. 

\subsection{Healthcare trackers, wearables and sensors}
They're fantastic tools for learning more about ourselves and regaining control of our life. There is a device for all of these needs and more! Whether you want to better manage your weight, stress, cognitive abilities, or overall fitness and energy, there is a device for all of these needs and more! The beauty of these new technology-driven gadgets is that they truly put patients at the centre of care. These technologies enable people to take charge of their health and make more informed decisions by allowing them to monitor their health at home and communicate the data with their physician remotely.
\subsection{3D-printing}
3D printing has the potential to revolutionise healthcare in every way. We can now print biotissues, artificial limbs, medications, blood vessels, and the list goes on, and it is likely that we will continue to do so in the future. Rensselaer Polytechnic Institute in Troy, New York, found a technology for printing living skin and blood vessels in 3D. This breakthrough is critical for burn victims who require skin grafts. This technique also benefits the pharmaceutical business. Since 2015, the FDA has approved 3D-printed medications, and researchers are now working on 3D-printed "polypills." These contain multiple layers of medications to aid patients in sticking to their treatment plans.
\end{document}